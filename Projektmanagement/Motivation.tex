\documentclass[12pt,a4paper]{scrreprt}
\usepackage[utf8]{inputenc}
\usepackage[german]{babel}
\usepackage[T1]{fontenc}
\usepackage{amssymb}
\usepackage{graphicx}
\usepackage{enumitem}
\usepackage[onehalfspacing]{setspace}


%--- PDF Dokumenteninformation ---
\usepackage[
		pdftitle={Sanfilippo - Mitarbeiter-Motivation},
		pdfsubject={},
		pdfauthor={Martin Sanfilippo},
		pdfkeywords={},
		hidelinks
]{hyperref}

%--- Besondere Wort-Trennung ---
\hyphenation{De-zi-mal-tren-nung}

%--- Quellenverzeichnis ---
\bibliographystyle{apalike}

%--- Abkürzungsverzeichnis (nur genutzte anzeigen) ---
\usepackage[printonlyused]{acronym}

%--- schicke Kopf- und Fusszeile ---
\usepackage{fancyhdr}
\usepackage[left = 2cm,right = 2cm,top = 3cm,bottom = 3cm]{geometry}

\pagestyle{fancy}
\lhead{\slshape Martin Sanfilippo - 862013}
\chead{}
\rhead{\slshape Mitarbeiter-Motivation}

\lfoot{}
\cfoot{\thepage}
\rfoot{}

\renewcommand{\headrulewidth}{0.2pt}
\renewcommand{\footrulewidth}{0pt}

%--- Inhalt Titelblatt ---
\author{Martin Sanfilippo - 862013 \\[1,0cm] Betreuer: Professor Dr. Michael Syrjakow\\Beuth University of Applied Sciences \\ \\ \small Diese Arbeit kann im Rahmen des\\ \small Projektmanagementkurs veröffentlich werden.}
\title{Projektmanagement\\ Mitarbeiter-Motivation}

\begin{document}

%--- Textbereich --- 
\large
\maketitle
\tableofcontents
\Large
\part{Einleitung}
\chapter{Vorwort}
\thispagestyle{fancy}
Projekte sind meist selbst bei kleinen Zielen recht umfangreich. Dem Neuling passieren beim Einstieg in die Projektarbeit daher auch meist Fehler, die vermeidbar sind. Aber auch erfahrenen Projektleitern werden Grenzen durch große Firmenkonstrukte über ihm gesetzt, die Projekte an die Grenze des Machbaren führen. Aus meiner eigenen Erfahrung kenne ich viele Projekte die unnötigerweise Unmut bei den Mitarbeitern erzeugt haben oder sogar an unterschiedlichsten Gründen gescheitert sind. Denn hier wurden die Mitglieder oder Mitarbeiter, die später mit dem Produkt arbeiten, nicht mitgenommen. Dies führte zu Unmut aber auch einer geringeren Arbeitsleistung. Aus meiner Sicht ist das fatal, denn schlussendlich scheitern dann Projekte an Themen wie Kommunikation und Motivation. \\
Das sind weichen Faktoren, denen man rechtzeitig hätte begegnen können, wie zum Beispiel durch klare Kommunikationsregeln\\  \cite[S. 78]{Braumandl2009}. \\Aber wie soll das bei der Motivation funktionieren?

\chapter{Abstract}
\thispagestyle{fancy}
Unabhängig von der Größe eines Projektes gibt es Faktoren, die sie alle gemein haben. Diese grundlegenden Aspekte lassen sich in allgemeine Kategorien einteilen. Einer der wichtigsten ist, neben dem eigentlichen Projekt, der ausführende Mitarbeiter. An der Basis jedes Projektes und dessen Erfolg sind Mitarbeiter zu finden, die durch eine gute Kommunikation und Motivation an das Projekt zu binden sind und so zu dessen Erfolg beitragen. \\ 
Wie der Chaos Report regelmäßig feststellt, gehört die Motivation von Mitarbeitern zu den fünf wichtigsten Punkten eines Projektes \cite{Chaosreport2014}. \\
Wenn die Projektleitung zu sehr auf die Werkzeuge fokussiert ist und die Mitarbeiter und deren Motivation aus den Augen verliert, scheitern unnötig viele IT-Service-Management-Projekte (ITSM). "52 Prozent der ITSM-Projekte scheitern aufgrund von ABC (Anm.: \textbf{A}ttitude, \textbf{B}ehaviour, \textbf{C}ulture)"\, \cite{Forrester2006}. \\[0,3cm]
Bei der Motivation der Mitarbeiter ist darauf zu achten, dass jeder seine persönliche Mischung von intrinsischer und extrinsischer Motivation mitbringt. Diese Motivation gilt es gezielt und auf den Einzelnen abgestimmt anzusprechen, was häufig eine Kombination von mehreren verstärkenden Faktoren ist. Wichtig ist dabei das Gleichgewicht zwischen den Mitarbeitern zu finden, damit die Arbeitsmoral und -leistung nicht wegen einer gefühlten Ungleichbehandlung leidet \cite{Adams1965}.\\
Mitarbeiter brauchen optimale Bedingungen, um ihre Aufgabe im Sinne der Organisation erfolgreich übernehmen zu können. Sie sind komplexe Persönlichkeiten mit ausgeprägten Fähigkeiten und Gestaltungskraft. Daher werden sie nicht nur angeführt und geleitet, sondern nehmen selbst Einfluss auf die Arbeitssituation und verändern damit laufend die Voraussetzung für die Führung durch den Vorgesetzten \cite{Lippmann2013}. \\

\part{Hauptteil}

\chapter{Mitarbeiter-Motivation}
\thispagestyle{fancy}
Um als Führungskraft gewinnbringenden Einfluss auf den Erfolg eines Projektes zu haben, ist es wichtig die Mitarbeiter nicht nur als Einheit wahrzunehmen, sondern in erster Linie als Individuen. Jeder Mitarbeiter hat eigene Interessen und Bedürfnisse. Diese spiegeln sich in seiner Arbeitsweise wider oder sind bei Nicht-Erfüllung unter anderem ein Grund, warum Projekte scheitern oder weniger erfolgreich verlaufen. Um diesen Interessen und Bedürfnissen begegnen und sie im Sinne des Projektes nutzen zu können, bedarf es einer Einschätzung der einzelnen Mitarbeiter. Denn nicht jeder Mitarbeiter reagiert auf dasselbe Arbeitsklima oder dieselben Maßnahmen der Führungskraft gleich. Dieser Fakt ist immer wieder neu zu bewerten, denn die jeweilige Lebenssituation kann immer wieder neu Einfluss darauf nehmen. Motivation lässt sich also definieren als die ''aktivierende Ausrichtung des momentanen Lebensvollzugs auf einen positiv bewerteten Zielzustand'' \cite[S. 17]{Rheinberg2002}.

\section{Motivationstheorien}
Als erstes soll an dieser Stelle die Motivation kurz definiert werden. Ein Motiv ist ein Grund, etwas zu tun. Motivation befasst sich daher mit den Einflussfaktoren, die Menschen zu einem bestimmten Verhalten bewegen. Motivation kann als zielgerichtetes Verhalten verstanden werden. Jemand ist motiviert, wenn er als Ergebnis bestimmter Handlungen die Erreichung eines bestimmten Ziels erwartet \cite{Armstrong1999}.\\
Die intrinsische und extrinsische Faktoren beschreiben die beiden grundlegenden Arten von Motivation und sollen daher im nächsten Kapitel genauer beleuchtet werden.

\subsection{Begrifflichkeiten}
Die Motivation eines Mitarbeiters ist immer in ihm selbst zu finden. Nicht jeder ist sich dessen selbst bewusst. Um so schwerer ist es für die Führungskraft oder den Projektleiter jeden Mitarbeiter mit der für ihn effektivsten Methode der Motivation anzusprechen.\\[0,3cm]
Grundsätzlich lassen sich alle Gründe der Motivation in erster Ebene in zwei Gruppen einteilen. Hier steht die intrinsische der extrinsischen Motivation gegenüber. Dies schließt aber nicht die Interaktion dieser beiden Systeme aus \cite{Heckhausen2006}.  \\[2,5cm]
\textbf{Intrinsisch}
	\begin{itemize}
		\item kognitive Neugier 
		\item emotionaler Anreiz
		\item Erfolgserwartung
	\end{itemize}
\textbf{Extrinsisch}
	\begin{itemize}
		\item Verstärkung durch Belohnung 
		\item Verstärkung durch Zwang
	\end{itemize}	
Extrinsisch tritt sehr häufig als die einzige Motivation auf, die von unerfahrenen Führungskräften oder Projektleitern wahrgenommen wird. Hier bekommt der Mitarbeiter für seine gute Arbeit oder für einen Erfolg im Projekt eine Aufmerksamkeit. Neben der offensichtlichen finanziellen Belohnung sind hier aber auch andere Aspekte zu nennen. So ist ein ''Aufstieg'' in ein eigenes Büro zu nennen, genauso wie eine bessere Ausstattung des Arbeitsplatzes oder der Technik, welche vom Mitarbeiter genutzt wird.\\
Auf der anderen Seite gibt es im weitesten Sinne die Bestrafung. Hier kann der Status quo bereits eine Drohung sein, wenn diese nicht zufriedenstellend ist. Aber auch eine Abwertung im Finanziellen oder der Position kann hier als extrinsische Motivation genutzt werden. Weniger offensichtlicher ist der Zwang eines finanziellen Risikos der arbeitgebenden Firma. So kann durch ein gescheitertes Projekt der eigene Arbeitsplatz oder auch die ganze Firma in finanzielle Schieflage geraten. Diese Motivation, kurz vor dem Abgrund zu stehen, ist nachhaltig aber auf Dauer nicht förderlich.\\[0,3cm]
Intrinsisch orientierte Mitarbeiter sind durch eine spannende Aufgabe und dessen Inhalt motiviert. Dies wird von Führungskräften häufig nicht ernst oder wahrgenommen. Nicht selten wird dann ein anderer Grund beim Mitarbeiter vermutet, der aber nicht bekannt werden soll. Hier wird das Interesse an der Arbeit nicht wahrgenommen. ''Interesse wiederum ist als kognitiv-affektive Erfahrung definiert, die bei positiver Erlebnistönung die Aufmerksamkeit auf die Tätigkeit bzw. Aufgabe lenkt und fokussiert. Man möchte die Aktivität hier und jetzt gerade tun und hat Freude dabei'' \cite{Rheinberg2002}. \\
Aber nicht nur der Aufgabeninhalt kann motivierend wirken, sondern auch die Möglichkeit der Selbstverwirklichung. Hier findet der Mitarbeiter also Freude an seiner Arbeit, möchte sich daran weiterentwickeln und etwas schaffen, auf das er dann positiv zurückblicken kann \cite{Rheinberg2002}.\\
Gerade hier ist es aber entscheidend, dass der Gegenstand der Arbeit als wertvoll erachtet und im Unternehmen anerkannt wird. Ist der Gegenstand nicht hoch klassifiziert, sinkt entsprechend die intrinsische Motivation.

\subsection{Erfahrungen}
Als Projektleiter sollte man sich aber zu einfach machen und versuchen den einen Faktor bei jedem Mitarbeiter zu finden, welcher motivierend wirkt. Denn hier handelt es sich immer um eine Kombination verschiedener Dinge \cite{Heckhausen2006}. \\
Denn eine extrinsische Motivation nach mehr Geld schließt keine weitere intrinsische Motivation beim Einzelnen aus.\\[0,3cm]
Darüber hinaus liefert das Herzberg Modell eine gute Erklärung, dass finanzielle Anreize (extrinsisch) allein nur bedingt zur Motivation und Zufriedenheit beitragen. Laut Herzberg ist die Bezahlung ein Hygienefaktor. Es wird erwartet, ausreichend gut für die eigene Arbeit bezahlt zu werden \cite{Herzberg1959}.

\subsection{Konflikte}
Wenn Mitarbeiter in einem Projekt unterschiedlich angesprochen und motiviert werden sollen, dann ist darauf zu achten, dass die Gleichheitstheorie nach Adams dennoch eingehalten wird. Die Theorie basiert auf der wahrgenommenen Behandlung der Mitarbeiter im Vergleich mit Kollegen. Hier wird keine identische Behandlung erwartet, sondern eben eine gleiche, faire Behandlung in Relation zu einer anderen Person oder Gruppe. Wird die Vergleichsperson, welche selbst gewählt wird, und der Mitarbeiter fair behandelt, dann ist dieser höher motiviert. Auch wenn die Gleichbehandlung nur ein Aspekt der Motivation ist, kann er einen erheblichen Einfluss auf die Arbeitsmoral und -leistung haben \cite{Adams1965}.

\section{Führung}
Als Führungskraft, aber auch als Projektleiter, ist es wichtig zu wissen, dass es keine richtige oder falsche Art von Führung gibt. Es ist wichtig, wie ein Klaviervirtuose über die große Klaviatur der Verhaltensweisen und Instrumentarien zu gleiten, um der komplexen Vielfalt und Dynamik der Organisationsrealität gewachsen zu sein \cite{Lippmann2013}. \\
Im Alltag des Projektes und auch bei der Auswahl der Motivationswerkzeuge gegenüber den Mitarbeitern bedarf es einer Führungsintelligenz. Diese bezeichnet die gelungene Verbindung  von kognitiver mit emotionaler Intelligenz. Die Fähigkeit, Einflüsse und Zusammenhänge zu sehen und gleichzeitig die zugehörigen Gefühle der eigenen Person und Dritter wahrzunehmen und diese Wahrnehmung bewusst in das eigene Verhalten einzubeziehen \cite{Lippmann2013}.

\subsection{Individuen}
Nach Maslow gibt es fünf grundlegende und aufeinander aufbauende Kategorien von Bedürfnissen. Beginnend mit physiologischen Bedürfnissen, über Sicherheits- und soziale Bedürfnisse, zu Achtung und Selbstverwirklichung. Maslow nimmt immer an, dass, wenn ein Bedürfnis erfüllt ist, automatisch danach das nächsthöhere Bedürfnis die treibende Rolle im Handeln des Menschen einnimmt. Ist ein niedriger bewertetes Bedürfnis erfüllt, so nimmt seine Wirkung als Motivationsfaktor für den Einzelnen ab.\\
Dieses Modell erklärt, warum allein die Möglichkeit zur Befriedigung grundlegender Bedürfnisse nach Nahrung und Kleidung kaum geeignet ist, einen Mitarbeiter im Sozialstaat Deutschland zu motivieren. Erst wenn die Arbeit die Befriedigung zusätzlicher Bedürfnisse in Aussicht stellt, besteht ein Anreiz. 

\subsection{Einfluss des Mitarbeiter}
Mitarbeiter brauchen optimale Bedingungen, um ihre Aufgabe im Sinne der Organisation erfolgreich übernehmen zu können. Sie sind komplexe Persönlichkeiten mit ausgeprägten Fähigkeiten und Gestaltungskraft. Daher werden sie nicht nur angeführt und geleitet, sondern nehmen selbst Einfluss auf die Arbeitssituation und verändern damit laufend die Voraussetzung für die Führung durch den Vorgesetzten \cite{Lippmann2013}. \\
Einfluss hat der Mitarbeiter hierdurch auch auf das Erreichen von Zielen. Daher sollte der Projektleiter diese sorgsam auswählen. Denn ihm obliegt es, durch überschaubare Sprints oder durch die Nutzung von Methoden wie Scrum, den Mitarbeitern erreichbare Ziele zu ermöglichen. Durch das immer wieder erfolgreiche Erreichen von Zielen steigt bei den einzelnen Mitarbeitern die intrinsische Motivation und kann so das Projekt positiv beeinflussen. 

\section{Ausblick}
Die Motivation an der Arbeit eines Projektes unterliegt nicht nur verhältnismäßig kurzen Veränderungsphasen, die ein Mensch im Laufe seines Lebens durchschreitet. Auch längerfristig gibt es Änderungen in der Motivation, deren Anfänge gerade erst erforscht sind. \\  
Nicht nur die Digitalisierung unseres Alltags, sondern im Speziellen auch die Digitalisierung unserer Arbeit, nimmt großen Einfluss. So kann heute bereits vieles automatisiert oder von weniger hoch qualifizierten Kräften bearbeitet werden, die nur noch für klar abgegrenzte Arbeiten hinzugezogen werden. An dieser Stelle greift keine langfristige Motivation, da die Mitarbeiter nur kurzfristig zur Verfügung stehen. Hierdurch wird sich das Gefüge in einem Projektteam verändern und auch unter diesen neuen Aspekten muss die Motivation neu betrachtet werden.


%--- Für Quellen ist das Programm JabRef nötig ---
\renewcommand{\bibname}{Quellen}
\addcontentsline{toc}{chapter}{Quellen}
\bibliography{Literatur}
\thispagestyle{fancy}


\end{document}